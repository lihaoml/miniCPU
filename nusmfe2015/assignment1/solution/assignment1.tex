
\documentclass[12pt,a4paper,hidelinks,fleqn]{article}            % Article 12pt font for a4 paper while hiding links
\usepackage[margin=1in]{geometry}                          % Required to adjust margins

%----------------------------------------------------------------------------------------
%    TYPE SETTING PACKAGES
%----------------------------------------------------------------------------------------

\usepackage[english]{babel}                                % English language/hyphenation 
\usepackage[utf8x]{inputenc}                               % Accept different input encodings
\usepackage{amsmath,amsfonts,amsthm,amssymb}               % Math packages to use equations
\usepackage{siunitx}                                       % Scientific units and numbering
\usepackage[usenames,dvipsnames,svgnames,table]{xcolor}    % Set color of text/background
\linespread{1.1}                                           % Default line spacing size
\usepackage{microtype}                                     % Improves spacing in the document
\usepackage{setspace}                                      % Set line spacing dynamically
\usepackage{tocloft}                                       % List adjustments including ToC
\usepackage{pgfplots, pgfplotstable}
%\usepackage{framed}
\usepackage[nocut]{thmbox}
\usepackage{footnote}
\usepackage{tablefootnote}
\usepackage{enumitem}

%----------------------------------------------------------------------------------------
%    FIGURES
%----------------------------------------------------------------------------------------

\usepackage{graphicx}                                      % Required for the inclusion of images
\graphicspath{{./Pictures/}}                               % Specifies picture directory
\usepackage{float}                                         % Allows putting an [H] in \begin{figure}
\usepackage{wrapfig}                                       % Allows in-line images
\usepackage{enumitem}
\usepackage[]{chapterbib}
\usepackage[titlenumbered,ruled]{algorithm2e}
\usepackage{verbatim}

\usepackage{hyperref}                                      % References
\usepackage{cleveref}                                      % Better References
%\crefname{lstlisting}{listing}{listings}
%\Crefname{lstlisting}{Listing}{Listings}
\crefname{figure}{figure}{figures}
\Crefname{figure}{Figure}{Figures}

%%% LINKS for ToC
\usepackage{hyperref}
\hypersetup{
   colorlinks,
   citecolor=black,
   filecolor=black,
   linkcolor=black,
   urlcolor=black
}

% make table at top even on an empty page
\makeatletter
    \setlength\@fptop{0\p@}
\makeatother
% to make nice multi column table
\usepackage{tabularx,booktabs}
\newcolumntype{Y}{>{\centering\arraybackslash}X}
\newcolumntype{s}{>{\hsize=.5\hsize}X}

%----------------------------------------------------------------------------------------
%    INCLUDE CODE
%----------------------------------------------------------------------------------------

\usepackage{listings}                                      % Package so code looks pretty
\lstset{
language=C,                                                % Choose the language
basicstyle=\footnotesize,                                  % The size of the fonts used
numbers=left,                                              % Where to put the line-numbers
numberstyle=\footnotesize,                                 % The size of the line-numbers
stepnumber=1,                                              % The step line-numbers
numbersep=5pt,                                             % How far the line-numbers are from the code
backgroundcolor=\color{white},                             % Choose the background color
showspaces=false,                                          % Show spaces adding partiular underscores
showstringspaces=false,                                    % Underline spaces within strings
showtabs=false,                                            % Show tabs within strings adding particular underscores
frame=single,                                              % Adds a frame around the code
tabsize=2,                                                 % Sets default tabsize to 2 spaces
captionpos=b,                                              % Sets the caption-position to bottom
breaklines=true,                                           % Sets automatic line breaking
breakatwhitespace=false,                                   % Sets if automatic breaks should only happen at whitespace
escapeinside={\%*}{*)}                                     % If you want to add a comment within your code
}

%----------------------------------------------------------------------------------------
%    COMMANDS
%----------------------------------------------------------------------------------------

\setlength\parindent{0pt}                                  % Removes all indentation from paragraphs
\setlength{\parskip}{4.0ex plus 0.5ex minus 0.3ex}         % Spacing between paragraphs
\renewcommand*\thesection{\arabic{section}}                % Renew section numbers
\renewcommand{\labelenumi}{\alph{enumi}.}                  % Section ordered numbering
\let\oldvec\vec                                            % Save the old vector style
\renewcommand{\vec}[1]{\boldsymbol{\mathbf{#1}}}
\DeclareMathOperator{\Tr}{Tr}                        % Set vectors to look like vectors
\renewcommand{\sfdefault}{phv}                             % Change default font
\renewcommand{\familydefault}{\sfdefault}                  % Use default font everywhere
\newcommand{\E}{\mathbb{E}}
\newcommand{\probP}{\mathbb{P}}
\newcommand{\probQ}{\mathbb{Q}}
\newcommand*{\Ito}{It\^{o} }
\newenvironment{concept}[1]
    {\vspace{5mm}
    \begin{thmbox}[L]{\textbf{#1}}
    }
    { 
    \end{thmbox}
    }
\makeatletter
\newenvironment{algoalign}
  {\setlength{\mathindent}{0pt}
   \vspace{-3mm}
   \start@align\@ne\st@rredtrue\m@ne
  }
  {\endalign
  \vspace{-8mm}
  }
\makeatother

\newcommand\numberthis{\addtocounter{equation}{1}\tag{\theequation}}


\newcommand{\argmax}{\operatornamewithlimits{argmax}}
\title{\vspace{-5ex}Assignment 1, FE5116 (2014/2015, Semester 2)\vspace{-7ex}}
\date{}
\begin{document}
\maketitle
\subsection*{Assignment 1a. Eight queens puzzle}
Write a program to place 8 chess queens on an 8x8 chessboard 
such that none of the queens is able to threaten each other.
That is, none of the queens share the same row, column or diagonal.
The solution is not unique, but your program is required to find only one valid solution.

The detailed description of the problem and possible algorithm to solve the problem can be found on wikipedia \url{http://en.wikipedia.org/wiki/Eight_queens_puzzle}.

Below is a recursive algorithm to tackle the problem.
It places the queens column by column incrementally. 
At placing column $i$, it picks a safe row against the previous placed columns ($1$ to $i-1$), 
and proceed to place the remaining columns using recursion. 
The way to check if a row $r$ is safe against the previously placed queen at ($r_p$, $c_p$) is to check if they are in the same row ($r = r_p$), the same column (by definition because we check only the previous columns), or the same diagonal ($r - i = r_p - r_p$ or $r + i = r_p + r_p$).
A solution is found if safe rows can be found all the way till the last column.
Otherwise, the algorithm rolls back and picks the next safe row at column i.
Below is the pseudo code of this algorithm.

\begin{algorithm}
	\caption{succeed = PlaceQ (prevQ, $i$)}
    \SetKwInOut{Input}{Input}
    \SetKwInOut{Output}{Output}
    \Input{prevQ: one dimensional array of length $i-1$, indicating the queens before column i are placed at (prevQ(k), k) for k $\in [1, i-1]$}
    \Output{succeed: whether it is possible to place the queens}
	\If{ $ i > 8 $} {
		solution found, print it out \\
		succeed = true;
	}
    \Else{ 
    	succeed = false; \\
    	\For{$r \gets 1$ \textbf{to} $8$} {
		  \If{ placing a queen at $(r, i)$ does not threaten prevQ} {
			\If {PlaceQ(prevQ + $(r, i)$, $i+1$) == true} {
		        succeed = true; \\
		        break; \# we are interested only in 1 solution so stop if we find one.
		        }		  
		  }
		}   
    }
\end{algorithm}

You can follow this algorithm or design your own algorithm.
But you should not brute force the problem.
The output of your program should be in this format:
\begin{verbatim}
   1   0   0   0   0   0   0   0
   0   0   0   0   1   0   0   0
   0   0   0   0   0   0   0   1
   0   0   0   0   0   1   0   0
   0   0   1   0   0   0   0   0
   0   0   0   0   0   0   1   0
   0   1   0   0   0   0   0   0
   0   0   0   1   0   0   0   0
\end{verbatim}
where $1$ represents the positions of the queens.

\subsection*{Assignment 1b. Approximation of $\pi$}
There are many infinite series for approximating $\pi$.
Implement three functions 
\begin{itemize}
\item myPi1($nTerm$): $\displaystyle \pi = 4 \sum_{k=0}^{nTerm} (-1)^{k}\frac{1}{2k+1}$
\item myPi2($nTerm$): $\displaystyle \pi = 3 + 4 \times \sum_{k=1}^{nTerm}(-1)^{k-1}\frac{1}{2k \times (2k+1) \times (2k+2)}$
\item myPi3($nTerm$): 
\begin{align*}
\pi = 4(4\arctan \frac{1}{5} - \arctan \frac{1}{239}), ~~~\text{where} 
~~~\arctan x = \sum_{k=0}^{nTerm} \frac{(-1)^k} {2k+1} x^{2k+1}
\end{align*}
\end{itemize}
and plot their convergence chart with X-axis the number of terms used $nTerm$, and Y-axis the error.

\subsection*{Assignment 1c. Decimal to Ternary conversion}
Modify the example function \verb=decInt2BinInt(x, nbits)= given in the lecture to  
a function that converts decimal integer to ternary integer (base-3): \verb=decInt2TerInt(x, nbits)=.

\end{document}
