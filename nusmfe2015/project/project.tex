\documentclass[11pt]{beamer}
\usetheme{CambridgeUS}
\usepackage[utf8]{inputenc}
\usepackage{amsmath}
\usepackage{amsfonts}
\usepackage{amssymb}
%\author{}
\title{Project, FE5116 (2014/2015, Semester 2)}
%\setbeamercovered{transparent} 
%\setbeamertemplate{navigation symbols}{} 
%\logo{} 
%\institute{} 
\date{12 March, 2015} 
%\subject{} 
\begin{document}

\begin{frame}
\titlepage
\end{frame}

%\begin{frame}
%\tableofcontents
%\end{frame}

\begin{frame}{Commodities Market}

\begin{itemize}
\item In commodities markets, spot or physical products are not directly tradeable in the exchange.
\item Instead, futures contracts that delivers the underlying commodity asset at different maturities are traded and observable from the exchange.
\item Usually, the movements of the short term futures contracts are large, while the movements of the long term futures contracts are relatively small.
\item This corresponds to a decreasing profile of the futures contracts' volatility with respect to the time to maturity
\end{itemize}

\end{frame}

\begin{frame}{Gabillon Two Factor Model}
A frequently used pricing model in commodities derivatives is Gabillon 2 factor model, named after Jacques Gabillon who proposed to model the term structure of commodities futures prices using a short term factor and a long term factor \cite{gabillon1991, schwartz1997, Clewlow99valuingenergy}.
\begin{itemize}
\item The short factor drives the dynamics of the short maturity futures contracts, and
\item the long factor drives the dynamics of the long maturity contracts.
\end{itemize}

\end{frame}

\begin{frame}{Gabillon Two Factor Model - Forward Dynamics}
The model can be stated in terms of forward dynamics:
\begin{align}
\frac{dF(t, T)}{F(t, T)} = \sigma_L (1 - e^{-\alpha(T-t)}) dW_L(t) + \sigma_S e^{-\alpha(T-t)} dW_s(t)
\label{eq:gab2-diffusion}
\end{align}
where 
\begin{itemize}
\item $F(t, T)$ represents the price at time $t$ of the futures contract that delivers the underlying commodity asset at time $T$,
\item $W_S$ and $W_L$ are the two Brownian motions that drives the short term and long term factors and they are correlated:
\begin{align*}
dW_S dW_L = \rho dt,
\end{align*}
\item $\sigma_S$ is the volatility of the short term factor
\item $\sigma_L$ is the volatility of the long term factor
\item $\alpha$ is the mean reversion speed. When $\alpha = 0$ the model reduced to a single factor spot model. 
\end{itemize}
\end{frame}


\begin{frame}{Gabillon Two Factor Model}
\begin{itemize}
\item The parameters of Gabillon two factor model are: $\sigma_S$, $\sigma_L$, $\alpha$, and $\rho$
\item For simplicity we assume all these parameters are constant.
\item One way to calibrate the model parameters is to match the model covariance matrix of the log returns to the historical covariance matrix of log returns for frequently traded futures maturities.
\end{itemize}

\end{frame}

\begin{frame}{Gabillon Two Factor Model - Model Covariance}
Applying Ito to \eqref{eq:gab2-diffusion} we can derive the diffusion of the log prices:
\begin{align*}
d \ln F(t, T) & = - \frac{1}{2}[ \sigma_S^2 e^{-2\alpha(T-t)} + \sigma_L^2(1-e^{-\alpha(T-t)})^2 \\
              & + 2 \rho \sigma_S \sigma_L \left(e^{-\alpha(T-t)} - e^{-2\alpha(T-t)}\right)] dt\\
              & + \sigma_L(1-e^{-\alpha(T-t)}) dW_L(t) 
                + \sigma_S e^{-\alpha(T-t)} dW_S(t)
\end{align*}

\end{frame}

\begin{frame}{Gabillon Two Factor Model - Model Covariance}
For futures contracts expiring at $T_1$ and $T_2$, we have
\begin{align*}
& d\ln F(t, T_1) d\ln F(t, T_2) \\
& = [\sigma_L^2(1-e^{-\alpha(T_1-t)} - e^{-\alpha(T_2-t)} + e^{-\alpha(T_1+T_2)+2\alpha t}) \\
& + \sigma_S^2 e^{-\alpha(T_1+T_2) + 2\alpha t} \\
& +\rho \sigma_S\sigma_L(e^{-\alpha(T_1-t)} + e^{-\alpha(T_2-t)} - 2e^{-\alpha(T_1+T_2)+2\alpha t}) ] dt
\end{align*}

\end{frame}

\begin{frame}{Gabillon Two Factor Model - Model Covariance}
Their covariance is 
\begin{align*}
&Cov\left(\ln\frac{F(t+\Delta_t, T_1)}{F(t, T_1)}, \ln\frac{F(t+\Delta_t, T_2)}{F(t, T_2)}\right) \\
&= \int_t^{t+\Delta t} d\ln F(t, T_1) d\ln F(t, T_2) \\
&= \sigma_L^2\Delta_t + (\rho\sigma_S\sigma_L) - \sigma_L^2)(e^{-\alpha T_1} + e^{-\alpha T_2})\left(\frac{e^{\alpha(t+\Delta_t)} - e^{\alpha t}}{\alpha}\right) \\
&~~+ (\sigma_S^2+\sigma_L^2 - 2\rho\sigma_S\sigma_L)e^{-\alpha(T_1+T_2)}\left(\frac{e^{2\alpha(t+\Delta_t)} - e^{2\alpha t}}{2\alpha}\right)
\end{align*}

\end{frame}

\begin{frame}{Calibration to Historical Covariance Matrix}
\begin{itemize}
\item Assuming we have the historical prices $P(t, t+T_i)$ of a commodity asset's future contracts for $i \in \{1, 2, ..., N\}$,
we can calculate the daily log returns
\begin{align*}
R(t, T_i) = \ln \frac{P(t+\Delta_t, t+\Delta_t + T_i)}{P(t, t+T_i)}.
\end{align*}
\item Let us denote $R_i$ as the time series of log returns for the contract maturity $T_i$.
We can then estimate the historical covariance $Cov_H (R_i, R_j)$.
\end{itemize}

\end{frame}

\begin{frame}{Calibration to Historical Covariance Matrix}
Now, to calibrate the Gabillon two factor model, we need to find the model parameters $\sigma_S, \sigma_L, \alpha, \rho$
such that the model covariance matches as close as possible to the estimated historical covariance matrix.

\begin{itemize}
\item The problem can be defined as finding $\sigma_S, \sigma_L, \rho, \alpha$ such that the objective function 
\begin{align*}
&Obj(\sigma_S, \sigma_L, \rho, \alpha) \\
&= \sum_{i=1}^{N} \sum_{j=1}^{N} \left(Cov\left(\ln\frac{F(\Delta_t, Ti)}{F(0, T_i)}, \ln\frac{F(\Delta_t, Tj)}{F(0, T_j)}\right) - Cov_H (R_i, R_j) \right)^2
\end{align*}
is minimized. 
\end{itemize}
\end{frame}

\begin{frame}{Calibration to Historical Covariance Matrix}
\begin{itemize}
\item The calibration is a non-linear optimization routine.
\item Octave provides a sequential quadratic programming solver for such kind of problem \url{https://www.gnu.org/software/octave/doc/interpreter/Nonlinear-Programming.html}
\item Be careful: initial guess is normally important to find an optimal solution.
\item constraints on the parameters:
\begin{itemize}
\item $\sigma_S > \sigma_L > 0$
\item $-1 \leq \rho \leq 1$
\item $\alpha > 0$
\end{itemize} 
\item We will then use the calibrated Gabillon two factor model to price Asian options
\end{itemize}
\end{frame}

\begin{frame}{Pricing Asian Options}
\begin{itemize}
\item Asian option is widely traded OTC product in commodity markets.
\item The payoff of arithmetic Asian call option is $(A - K)_+$ where $A = \frac{1}{n}\sum F(t_i, T_i)$.  
\item The expectation of an Asian call payoff is
\begin{align*}
E[(A - K)_+] = \int...\int(A-K)_+p(F(t_1, T_1),...,F(t_n, T_n))\Pi_i^n dF(t_i, T_i).
\end{align*}
We need to solve a multi-dimensional integral. Since the integrand is not separable, the expectation is numerically very difficult to solve for large n (for monthly average $n\approx 20$).
\end{itemize}
\end{frame}


\begin{frame}{Pricing Asian Options}
\begin{itemize}
\item The natural pricing method is Monte-Carlo.  
\item Two moment matching method (MM2, \cite{turnbull_wakeman1991_mm2}) is an alternative to price Asian option.
\item We know that $F(t_i, T_i)$ is log-normal variable.
But sum of lognormal variables is not lognormal.
\item The MM2 method approximates the distribution of $A$ using a lognormal distribution 
that matches the first two moments of $A$. 
\item Pros: fast (semi-analytic), stable price and greeks. 
\item Cons: error due to approximation (but negligible for most of the frequently trades products).

\end{itemize}
\end{frame}


\begin{frame}{Pricing Asian Option with Two Moment Matching}
Under the assumption of log-normality, we can use the Black-Scholes formula:
\begin{align*}
Call(A, K, t_s) = e^{-rt_s}[M N(d_+) - KN(d_-)]
\end{align*}
where
\begin{align*}
d_{\pm} = \frac{\ln (M/K) \pm \frac{1}{2}S^2}{S}, 
~M = E[A], ~S=\sqrt{Var[\ln A]}.
\end{align*}
and $N(.)$ is the standard norm cdf.
\end{frame}

\begin{frame}{Pricing Asian Option with Two Moment Matching}
The first moment of $A$ gives $M$:
\begin{align*}
M = E[A] = E[\frac{1}{n}\sum_{i=1}^n F(t_i, T_i)] = \frac{1}{n}\sum_{i=1}^n E[F(t_i, T_i)] = \frac{1}{n} \sum_{i=1}^n F(0, T_i)
\end{align*}
The second moment of $A$ gives $S$:
\begin{align*}
S^2 = Var[\ln A] = \ln \left( \frac{E[A^2]}{E[A]^2} \right)
\end{align*}
where
\begin{align}
E[A^2] = \frac{1}{n^2}\sum_{i=1}^n \sum_{j=1}^n E[F(t_i, T_i) F(t_j, T_j)] 
\label{eq:sec-moment}
\end{align}

\end{frame}

\begin{frame}{Project Requirements}
Design and implement a framework that 
\begin{itemize}
\item calibrates the Gabillon two factor model parameters to given historical prices,
\item price arithmetic Asian option with the calibrated Gabillon two factor model, using two moment matching method.
\end{itemize}

Input:
\begin{itemize}
\item the market maturities. A ``tenors.mat'' file will be provided that contains an size N array of $T_i$ (in years). For example [0.25, 0.5, 1, 2] means N = 4 and the market contains 3 month, 6 month, 1 year and 2 year contracts.
\item one year historical prices of the N futures contracts whose maturities are aligned with ``tenors.mat''.
A file ``hist.mat'' will store a 250 x N matrix, whose first row represents today's prices, second row represents yesterday's price, and so on. This can be used to calibrate the model parameters. In addition, today's price ($F(0, T)$) should be used for pricing 
\end{itemize}
\end{frame}

\begin{frame}{Project Submission}
The project submission package should contain
\begin{itemize}
\item All the source code for calibration and pricing
\item A runnable test script
\item A project report in PDF format that documents
\begin{itemize}
\item the design of the framework
\item the implementation details (explanation of the submitted files and functions)
\item the tests performed and test files (calibration errors, pricing Asian options with different expiries and averaging period, etc)
\item in particular, the initial guess and constraints used in the optimization routine and the completion of equation \eqref{eq:sec-moment} applying Gabillon two factor model.
\end{itemize}
\item Questions or clarifications can be posted on IVLE forum for discussion.
\end{itemize}
\end{frame}

\begin{frame}[allowframebreaks]
        \frametitle{References}
        \bibliographystyle{amsalpha}
        \bibliography{qftn}
\end{frame}
\end{document}

\subsubsection*{Pricing an Asian Option using Two Moment Matching}


