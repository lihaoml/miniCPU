\documentclass[12pt,a4paper,hidelinks,fleqn]{article}            % Article 12pt font for a4 paper while hiding links
\usepackage[margin=1in]{geometry}                          % Required to adjust margins

%----------------------------------------------------------------------------------------
%    TYPE SETTING PACKAGES
%----------------------------------------------------------------------------------------

\usepackage[english]{babel}                                % English language/hyphenation 
\usepackage[utf8x]{inputenc}                               % Accept different input encodings
\usepackage{amsmath,amsfonts,amsthm,amssymb}               % Math packages to use equations
\usepackage{siunitx}                                       % Scientific units and numbering
\usepackage[usenames,dvipsnames,svgnames,table]{xcolor}    % Set color of text/background
\linespread{1.1}                                           % Default line spacing size
\usepackage{microtype}                                     % Improves spacing in the document
\usepackage{setspace}                                      % Set line spacing dynamically
\usepackage{tocloft}                                       % List adjustments including ToC
\usepackage{pgfplots, pgfplotstable}
%\usepackage{framed}
\usepackage[nocut]{thmbox}
\usepackage{footnote}
\usepackage{tablefootnote}
\usepackage{enumitem}

%----------------------------------------------------------------------------------------
%    FIGURES
%----------------------------------------------------------------------------------------

\usepackage{graphicx}                                      % Required for the inclusion of images
\graphicspath{{./Pictures/}}                               % Specifies picture directory
\usepackage{float}                                         % Allows putting an [H] in \begin{figure}
\usepackage{wrapfig}                                       % Allows in-line images
\usepackage{enumitem}
\usepackage[]{chapterbib}
\usepackage[titlenumbered,ruled]{algorithm2e}
\usepackage{verbatim}

\usepackage{hyperref}                                      % References
\usepackage{cleveref}                                      % Better References
%\crefname{lstlisting}{listing}{listings}
%\Crefname{lstlisting}{Listing}{Listings}
\crefname{figure}{figure}{figures}
\Crefname{figure}{Figure}{Figures}

%%% LINKS for ToC
\usepackage{hyperref}
\hypersetup{
   colorlinks,
   citecolor=black,
   filecolor=black,
   linkcolor=black,
   urlcolor=black
}

% make table at top even on an empty page
\makeatletter
    \setlength\@fptop{0\p@}
\makeatother
% to make nice multi column table
\usepackage{tabularx,booktabs}
\newcolumntype{Y}{>{\centering\arraybackslash}X}
\newcolumntype{s}{>{\hsize=.5\hsize}X}

%----------------------------------------------------------------------------------------
%    INCLUDE CODE
%----------------------------------------------------------------------------------------

\usepackage{listings}                                      % Package so code looks pretty
\lstset{
language=C,                                                % Choose the language
basicstyle=\footnotesize,                                  % The size of the fonts used
numbers=left,                                              % Where to put the line-numbers
numberstyle=\footnotesize,                                 % The size of the line-numbers
stepnumber=1,                                              % The step line-numbers
numbersep=5pt,                                             % How far the line-numbers are from the code
backgroundcolor=\color{white},                             % Choose the background color
showspaces=false,                                          % Show spaces adding partiular underscores
showstringspaces=false,                                    % Underline spaces within strings
showtabs=false,                                            % Show tabs within strings adding particular underscores
frame=single,                                              % Adds a frame around the code
tabsize=2,                                                 % Sets default tabsize to 2 spaces
captionpos=b,                                              % Sets the caption-position to bottom
breaklines=true,                                           % Sets automatic line breaking
breakatwhitespace=false,                                   % Sets if automatic breaks should only happen at whitespace
escapeinside={\%*}{*)}                                     % If you want to add a comment within your code
}

%----------------------------------------------------------------------------------------
%    COMMANDS
%----------------------------------------------------------------------------------------

\setlength\parindent{0pt}                                  % Removes all indentation from paragraphs
\setlength{\parskip}{4.0ex plus 0.5ex minus 0.3ex}         % Spacing between paragraphs
\renewcommand*\thesection{\arabic{section}}                % Renew section numbers
\renewcommand{\labelenumi}{\alph{enumi}.}                  % Section ordered numbering
\let\oldvec\vec                                            % Save the old vector style
\renewcommand{\vec}[1]{\boldsymbol{\mathbf{#1}}}
\DeclareMathOperator{\Tr}{Tr}                        % Set vectors to look like vectors
\renewcommand{\sfdefault}{phv}                             % Change default font
\renewcommand{\familydefault}{\sfdefault}                  % Use default font everywhere
\newcommand{\E}{\mathbb{E}}
\newcommand{\probP}{\mathbb{P}}
\newcommand{\probQ}{\mathbb{Q}}
\newcommand*{\Ito}{It\^{o} }
\newenvironment{concept}[1]
    {\vspace{5mm}
    \begin{thmbox}[L]{\textbf{#1}}
    }
    { 
    \end{thmbox}
    }
\makeatletter
\newenvironment{algoalign}
  {\setlength{\mathindent}{0pt}
   \vspace{-3mm}
   \start@align\@ne\st@rredtrue\m@ne
  }
  {\endalign
  \vspace{-8mm}
  }
\makeatother

\newcommand\numberthis{\addtocounter{equation}{1}\tag{\theequation}}


\newcommand{\argmax}{\operatornamewithlimits{argmax}}
\title{\vspace{-5ex}Project, FE5116 (2014/2015, Semester 2)\vspace{-7ex}}
\date{}
\begin{document}
\maketitle
\paragraph{Model Parameter Calibration}

Gabillon model dynamics
\begin{align}
\frac{dF(t, T)}{F(t, T)} = \sigma_L (1 - e^{-\alpha(T-t)}) dW_L(t) + \sigma_S e^{-\alpha(T-t)} dW_s(t)
\label{eq:gab2-diffusion}
\end{align}
\begin{itemize}
\item $F(t, T)$ represents the price at time $t$ of the futures contract that delivers the underlying commodity asset at time $T$,
\item $W_S$ and $W_L$ are the two Brownian motions that drives the short term and long term factors and they are correlated:
\begin{align*}
dW_S dW_L = \rho dt,
\end{align*}
\item $\sigma_S$ is the volatility of the short term factor
\item $\sigma_L$ is the volatility of the long term factor
\item $\alpha$ is the mean reversion speed. When $\alpha = 0$ the model reduced to a single factor spot model. 
\end{itemize}

Applying Ito to \eqref{eq:gab2-diffusion} we can derive the diffusion of the log prices:
\begin{align*}
d \ln F(t, T) & = - \frac{1}{2}[ \sigma_S^2 e^{-2\alpha(T-t)} + \sigma_L^2(1-e^{-\alpha(T-t)})^2 \\
              & + 2 \rho \sigma_S \sigma_L \left(e^{-\alpha(T-t)} - e^{-2\alpha(T-t)}\right)] dt\\
              & + \sigma_L(1-e^{-\alpha(T-t)}) dW_L(t) 
                + \sigma_S e^{-\alpha(T-t)} dW_S(t)
\end{align*}
For futures contracts expiring at $T_1$ and $T_2$, we have
\begin{align*}
& d\ln F(t, T_1) d\ln F(t, T_2) \\
& = [\sigma_L^2(1-e^{-\alpha(T_1-t)} - e^{-\alpha(T_2-t)} + e^{-\alpha(T_1+T_2)+2\alpha t}) \\
& + \sigma_S^2 e^{-\alpha(T_1+T_2) + 2\alpha t} \\
& +\rho \sigma_S\sigma_L(e^{-\alpha(T_1-t)} + e^{-\alpha(T_2-t)} - 2e^{-\alpha(T_1+T_2)+2\alpha t}) ] dt
\end{align*}

The model covariance is
\begin{align}
&Cov\left(\ln\frac{F(t+\Delta_t, T_1)}{F(t, T_1)}, \ln\frac{F(t+\Delta_t, T_2)}{F(t, T_2)}\right) \\
&= \int_t^{t+\Delta t} d\ln F(t, T_1) d\ln F(t, T_2) \\
&= \sigma_L^2\Delta_t + (\rho\sigma_S\sigma_L - \sigma_L^2)(e^{-\alpha T_1} + e^{-\alpha T_2})\left(\frac{e^{\alpha(t+\Delta_t)} - e^{\alpha t}}{\alpha}\right) \\
&~~+ (\sigma_S^2+\sigma_L^2 - 2\rho\sigma_S\sigma_L)e^{-\alpha(T_1+T_2)}\left(\frac{e^{2\alpha(t+\Delta_t)} - e^{2\alpha t}}{2\alpha}\right)
\end{align}

The calibration problem is defined as finding $\sigma_S, \sigma_L, \rho, \alpha$ such that the objective function 
\begin{align}
&Obj(\sigma_S, \sigma_L, \rho, \alpha) \\
&= \sum_{i=1}^{N} \sum_{j=1}^{N} \left(Cov\left(\ln\frac{F(\Delta_t, Ti)}{F(0, T_i)}, \ln\frac{F(\Delta_t, Tj)}{F(0, T_j)}\right) - Cov_H (R_i, R_j) \right)^2
\end{align}
is minimized. 

The historical data in \verb=hist.dat= can be used to generate the historical covariance matrix.
\begin{itemize}
\item The daily log returns
\begin{align*}
R(t, T_i) = \ln \frac{P(t+\Delta_t, t+\Delta_t + T_i)}{P(t, t+T_i)}.
\end{align*}
\item Let us denote $R_i$ as the time series of log returns for the contract maturity $T_i$.
We can then estimate the historical covariance $Cov_H (R_i, R_j)$.
\end{itemize}
\begin{verbatim}
function histmat = histCovar(hist)
  # daily log return series
  rets = zeros (rows(hist) -1, columns(hist));
  for i = 1:rows(hist)-1
    rets(i, :) = log(hist(i, :) ./ hist(i+1, :));
  endfor
  histmat = rets' * rets ./ (rows(hist)-1);
endfunction
\end{verbatim}

Now, to calibrate the Gabillon two factor model, we need to find the model parameters $\sigma_S, \sigma_L, \alpha, \rho$
such that the model covariance matches as close as possible to the estimated historical covariance matrix.

So we first construct the model covariance function:
\begin{small}
\begin{verbatim}
function c = modelCovar(sigmaL, sigmaS, rho, alpha, dt, T1, T2)
  sigmaL2 = sigmaL * sigmaL;
  rSL = rho * sigmaS * sigmaL; 
  e1 = exp (-alpha * T1);
  e2 = exp (-alpha * T2);
  e3 = exp (alpha * dt);
  c = sigmaL2 * dt + (rSL - sigmaL2) * (e1 + e2) * (e3 - 1) / alpha \
  + (sigmaS * sigmaS + sigmaL2 - 2 * rSL) * (e1 * e2) * (e3 * e3 -1) / alpha / 2.0;
endfunction
\end{verbatim}
\end{small}

Then the objective function 
\begin{verbatim}
function err = objFun(histmat, dt, tenors, sigmaL, sigmaS, rho, alpha)
  err = 0;
  for i = 1:length(tenors)
    for j = 1:length(tenors)
      diff = modelCovar(sigmaL, sigmaS, rho, alpha, dt, tenors(i), tenors(j)) - histmat(i, j);
      err += diff * diff;
    endfor
  endfor
endfunction
\end{verbatim}
And the calibration main function:
\begin{verbatim}
function [sigmaL, sigmaS, rho, alpha] = gab2Calib(hist, tenors)
  histmat = histCovar(hist, tenors)
  dt = 1/250;
  f = @(params) objFun(histmat, dt, tenors, params(1), params(2), params(3), params(4));
  sigmaL_init = sqrt(histmat(length(tenors), length(tenors)) / dt)
  sigmaS_init = sqrt(histmat(1, 1) / dt)
  rho_init = histmat(1, length(tenors))  / dt / sigmaL_init / sigmaS_init
  init = [ sigmaL_init, sigmaS_init, rho_init, 0.5]

  [x] = sqp (init, f, [], @constraint);
  x
  f(x)
endfunction
\end{verbatim}
Note that the initial guess are based on the volatility of the front and end contracts (short and long).
The constraints are defined as
\begin{verbatim}
function r = constraint(params)
  r = [ params(3) + 1; 1 - params(3);  # -1 <= rho <= 1
        params(2); params(1); params(2)-params(1); # sigmaS > sigmaL > 0
        params(4)];  # alpha > 0
endfunction
\end{verbatim}

\paragraph{Pricing Asian Option using MM2}
The payoff of arithmetic Asian call option is $(A - K)_+$ where $A = \frac{1}{n}\sum F(t_i, T_i)$.  

Under the assumption of log-normality, we can use the Black-Scholes formula:
\begin{align*}
Call(A, K, t_s) = e^{-rt_s}[M N(d_+) - KN(d_-)]
\end{align*}
where
\begin{align*}
d_{\pm} = \frac{\ln (M/K) \pm \frac{1}{2}S^2}{S}, 
~M = E[A], ~S=\sqrt{Var[\ln A]}.
\end{align*}
and $N(.)$ is the standard norm cdf.

The first moment of $A$ gives $M$:
\begin{align*}
M = E[A] = E[\frac{1}{n}\sum_{i=1}^n F(t_i, T_i)] = \frac{1}{n}\sum_{i=1}^n E[F(t_i, T_i)] = \frac{1}{n} \sum_{i=1}^n F(0, T_i)
\end{align*}
The second moment of $A$ gives $S$:
\begin{align*}
S^2 = Var[\ln A] = \ln \left( \frac{E[A^2]}{E[A]^2} \right)
\end{align*}
where
\begin{align}
E[A^2] = \frac{1}{n^2}\sum_{i=1}^n \sum_{j=1}^n E[F(t_i, T_i) F(t_j, T_j)] 
\label{eq:sec-moment}
\end{align}

Recall that
\begin{align*}
d \ln F(t, T) & = \underbrace{- \frac{1}{2}[ \sigma_S^2 e^{-2\alpha(T-t)} + \sigma_L^2(1-e^{-\alpha(T-t)})^2  + 2 \rho \sigma_S \sigma_L \left(e^{-\alpha(T-t)} - e^{-2\alpha(T-t)}\right)]}_{\mu(t, T)} dt\\
              & + \underbrace{\sigma_L(1-e^{-\alpha(T-t)})}_{\sigma_1(t, T)} dW_L(t) 
                + \underbrace{\sigma_S e^{-\alpha(T-t)}}_{\sigma_2(t, T)} dW_S(t)
\end{align*}
so
\begin{small}
\begin{align*}
& F(t, T) = F(0, T) \exp\left(\int_0^t \mu(t, T) dt  + \int_0^t \sigma_1(t, T) dW_L(t) + \int_0^t \sigma_2(t, T) dW_S(t)\right) \\
&~~~\Downarrow \\
& F(t_i, T_j) F(t_j, T_j) = F(0, T_i) F(0, T_j) \exp\left( \int_0^{t_i} \mu(t, T_i) dt + \int_0^{t_j} \mu(t, T_j) dt \right)\times\\
&~~~~~~\exp\left( 
\int_0^{t_i} \sigma_1(t, T_i) dW_L(t) + \int_0^{t_i} \sigma_2(t, T_i) dW_S(t)
+ \int_0^{t_j} \sigma_1(t, T_j) dW_L(t) + \int_0^{t_j} \sigma_2(t, T_j) dW_S(t)
\right) \\
\end{align*}
without loss of generality, we assume $t_i < t_j$,
\begin{align*}
& F(t_i, T_j) F(t_j, T_j) = F(0, T_i) F(0, T_j) \exp\left( \int_0^{t_i} (\mu(t, T_i) + \mu(t, T_j)) dt \right)\times\\
&~~~~~~\underbrace{\exp\left( 
\int_0^{t_i} (\sigma_1(t, T_i) + \sigma_1(t, T_j)) dW_L(t) 
+ \int_0^{t_i} (\sigma_2(t, T_i) + \sigma_2(t, T_j)) dW_S(t) \right)}_{B} \times \\
&~~~~~~\underbrace{\exp\left( \int_{t_i}^{t_j} \mu(t, T_j) dt + 
\int_{t_i}^{t_j} \sigma_1(t, T_j) dW_L(t) + \int_{t_i}^{t_j} \sigma_2(t, T_j) dW_S(t)
\right)}_{C} \\
\end{align*}
Note that $B$ and $C$ are independent and $E[C] = 1$ since $\mu = -\frac12(\sigma_1^2 + \sigma_2^2 + 2\rho \sigma_1 \sigma_2)$, so
\begin{align*}
& E[BC] = E[B]E[C] = E[B] \\ 
& = \exp\left[\int_0^{t_i}\left( -\mu(t, T_i) - \mu(t, T_j) + \sigma_{1, i}(t)\sigma_{1, j}(t) + \sigma_{2, i}(t)\sigma_{2,j}(t) + \rho \sigma_{1, i}(t) \sigma_{2, j}(t) + \rho \sigma_{2, i} \sigma_{1, j}(t)\right) dt\right]
\end{align*}
Note that we simplified the notation $\sigma_{1, i}(t) = \sigma_1(t, T_i)$.
So 
\begin{align*}
E[F(t_i, T_j) F(t_j, T_j)] = & F(0, T_i) F(0, T_j) \times \\
	& \left[\int_0^{t_i}\left(\sigma_{1, i}(t)\sigma_{1, j}(t) + \sigma_{2, i}(t)\sigma_{2,j}(t) + \rho \sigma_{1, i}(t) \sigma_{2, j}(t) + \rho \sigma_{2, i} \sigma_{1, j}(t)\right) dt\right]
\end{align*}
\end{small}
And
\begin{align*}
\begin{cases}
\int_0^{t_i} \sigma_{1,i} (t) \sigma_{1, j} (t) dt = \sigma_L^2\left( 
t_i - \frac{e^{-\alpha T_i} + e^{-\alpha T_j}}{\alpha} (e^{\alpha t_i} - 1) + \frac{e^{-\alpha(T_i + T_j)}}{2\alpha} (e^{2\alpha t_i} - 1)
\right)\\
\int_0^{t_i} \sigma_{2,i} (t) \sigma_{2, j} (t) dt = \sigma_S^2 \left(\frac{e^{-\alpha(T_i + T_j)}}{2\alpha} (e^{2\alpha t_i} - 1)\right) \\
\int_0^{t_i} \rho \sigma_{1,i} (t) \sigma_{2, j} (t) dt = \rho \sigma_S \sigma_L \left( \frac{e^{-\alpha T_j}}{\alpha} (e^{\alpha t_i - 1}) - \frac{e^{ - \alpha(T_i + T_j)}}{2\alpha}(e^{2\alpha t_i} - 1)\right) \\
\int_0^{t_i} \rho \sigma_{2,i} (t) \sigma_{1, j} (t) dt = \rho \sigma_S \sigma_L \left( \frac{e^{-\alpha T_i}}{\alpha} (e^{\alpha t_i - 1}) - \frac{e^{ - \alpha(T_i + T_j)}}{2\alpha}(e^{2\alpha t_i} - 1)\right)\\
\end{cases}
\end{align*}

\begin{align*}
E[A^2] = \frac{1}{n^2}\sum_{i=1}^n \sum_{j=1}^n E[F(t_i, T_i) F(t_j, T_j)] 
\end{align*}

\paragraph{MM2 Source code}
\begin{verbatim}
function pv = mm2Gab2AsianCall(sigmaL, sigmaS, rho, alpha, priceCurve, ti_s, Ti_s, K, rf, T)
  m = sum(arrayfun(priceCurve, Ti_s)) / length(Ti_s)
  v = 0;
  for i = 1:length(Ti_s)
    for j = 1:length(Ti_s)
      v += expectation (sigmaL, sigmaS, rho, alpha, priceCurve, ti_s(i), ti_s(j), Ti_s(i), Ti_s(j));
    endfor
  endfor
  v = v / length(Ti_s) / length(Ti_s);
  var = v / m / m
  d1 = (log (m/K) + 0.5 * var) / sqrt(var);
  d2 = (log (m/K) - 0.5 * var) / sqrt(var);
  pv = exp(-rf*T) * (m * normcdf(d1) - K*normcdf(d2))
endfunction

function e = expectation (sigmaL, sigmaS, rho, alpha, priceCurve, ti, tj, Ti, Tj)
  t = min(ti, tj);
  a = sigmaL*sigmaL * (t - (exp(-alpha * Ti) + exp(-alpha * Tj))/alpha * (exp(alpha * t) -1) + exp(-alpha *(Ti+Tj))/2/alpha * (exp(2*alpha * t) -1) );
  b = sigmaS*sigmaS * exp(- alpha*(Ti + Tj)) / 2 / alpha * (exp(2*alpha * t) -1);
  c = rho * sigmaL * sigmaS * (exp(-alpha*Tj)/alpha * (exp(alpha * t) -1) - exp(-alpha*(Ti+Tj)) / alpha /2 * (exp(2*alpha * t) -1) );
  d = rho * sigmaL * sigmaS * (exp(-alpha*Ti)/alpha * (exp(alpha * t) -1) - exp(-alpha*(Ti+Tj)) / alpha /2 * (exp(2*alpha * t) -1) );
  e = a + b + c + d;
  e *= priceCurve(Ti) * priceCurve(Tj);
endfunction 
\end{verbatim}

\paragraph{Test case}
\begin{verbatim}
load("testData/tenors.mat")
load("testData/hist.mat")
[sigmaL, sigmaS, rho, alpha] = gab2Calib(hist, tenors)
priceCurve = @(t) interp1(tenors, hist(1, :), t);
sigmaL
ti_s = [0.1, 0.2, 0.3];
Ti_s = [1, 1, 1];
mm2Gab2AsianCall(sigmaL, sigmaS, rho, alpha, priceCurve, ti_s, Ti_s, 45, 0.01, 0.6)
mm2Gab2AsianCall(sigmaL, sigmaS, rho, alpha, priceCurve, ti_s, Ti_s, 50, 0.01, 0.6)
\end{verbatim}

We test against a degenerate case that $\rho = 1, \sigma_L = \sigma_S$, the model degenerates to Black-Scholes:
\begin{align*}
\frac{dF(t, T)}{F(t, T)} = \sigma_S dW_S
\end{align*}
Pricing a single observation Asian option \verb~ti_s = [1], Ti_s = [1], T = 1~ using MM2 should give the same price as Black-Scholes formula.
\begin{verbatim}
"degenerate case"
mm2Gab2AsianCall(sigmaS, sigmaS, 1.0, 0.1, priceCurve, [1], [1], 50, 0.01, 1)
# BS reference for degenerate case
var = sigmaS * sigmaS * 1.0
m = priceCurve(1)
K = 50;
rf = 0.01;
T = 1;
d1 = (log (m/K) + 0.5 * var) / sqrt(var);
d2 = (log (m/K) - 0.5 * var) / sqrt(var);
bspv = exp(-rf*T) * (m * normcdf(d1) - K*normcdf(d2))
\end{verbatim}
\end{document}

