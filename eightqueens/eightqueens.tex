
%----------------------------------------------------------------------------------------
%    PAGE ADJUSTMENTS
%----------------------------------------------------------------------------------------

\documentclass[12pt,a4paper,hidelinks,fleqn]{article}            % Article 12pt font for a4 paper while hiding links
\usepackage[margin=1in]{geometry}                          % Required to adjust margins
../styleAndCommands.tex
\title{\vspace{-3ex}Assignment 1\vspace{-7ex}}
\date{}
\begin{document}
\maketitle
\subsection*{Assignment 1a. Eight queens puzzle}
Write a program to place 8 chess queens on an 8x8 chessboard 
such that none of the queens is able to threaten each other.
That is, none of the queens share the same row, column or diagonal.
The solution is not unique, but your program is required to find only one valid solution.

The detailed description of the problem and possible algorithm to solve the problem can be found on wikipedia \url{http://en.wikipedia.org/wiki/Eight_queens_puzzle}.

Below is a recursive algorithm to tackle the problem.
It places the queens column by column incrementally. 
At placing column $i$, it picks a safe row against the previous placed columns ($1$ to $i-1$), 
and proceed to place the remaining columns. 
A solution is found if safe rows can be found all the way till the last column.
Otherwise, the algorithm rolls back and picks the next safe row at column i.
Below is the pseudo code of this algorithm.

\begin{algorithm}
	\caption{succeed = PlaceQ (prevQ, $i$)}
    \SetKwInOut{Input}{Input}
    \SetKwInOut{Output}{Output}
    \Input{prevQ: one dimensional array of length $i-1$, indicating the queens before column i are placed at (prevQ(k), k) for k $\in [1, i-1]$}
    \Output{succeed: whether it is possible to place the queens}
	\If{ $ i > 8 $} {
		solution found, print it out
	}
    \Else{ 
    	succeed = false; \\
    	\For{$r \gets 1$ \textbf{to} $8$} {
		  \If{ placing a queen at $(r, i)$ does not threaten prevQ} {
			\If {place(prevQ + $(r, i)$, $i+1$) == true} {
		        succeed = true; \\
		        break; \# we are interested only in 1 solution so stop if we find one.
		        }		  
		  }
		}   
    }
\end{algorithm}

You can follow this algorithm or design your own algorithm.
But you should not brute force the problem.
The output of your program should be in this format:
\begin{verbatim}
   1   0   0   0   0   0   0   0
   0   0   1   0   0   0   0   0
   0   0   0   0   0   1   0   0
   0   0   0   0   0   0   0   1
   0   0   0   0   0   0   1   0
   0   0   0   1   0   0   0   0
   0   1   0   0   0   0   0   0
   0   0   0   0   1   0   0   0
\end{verbatim}
where $1$ represents the positions of the queens.

\subsection*{Assignment 1b. Approximation of $\pi$}
There are many infinite series for approximating $\pi$.
Implement three functions 
\begin{itemize}
\item myPi1($nTerm$): $\displaystyle \pi = 4 \sum_{k=0}^{nTerm} (-1)^{k}\frac{1}{2k+1}$
\item myPi2($nTerm$): $\displaystyle \pi = 3 + 4 \times \sum_{k=1}^{nTerm}(-1)^{k-1}\frac{1}{2k \times (2k+1) \times (2k+2)}$
\item myPi3($nTerm$): 
\begin{align*}
\pi = 4(4\arctan \frac{1}{5} - \arctan \frac{1}{239}), ~~~\text{where} 
~~~\arctan x = \sum_{k=0}^{nTerm} \frac{(-1)^k} {2k+1} x^{2k+1}
\end{align*}
\end{itemize}
and plot their convergence chart with X-axis the number of terms used $nTerm$, and Y-axis the error.
\end{document}