
\documentclass[11pt,a4paper,hidelinks,fleqn]{article}            % Article 12pt font for a4 paper while hiding links
\usepackage[margin=1in]{geometry}                          % Required to adjust margins

%----------------------------------------------------------------------------------------
%    TYPE SETTING PACKAGES
%----------------------------------------------------------------------------------------

\usepackage[english]{babel}                                % English language/hyphenation 
\usepackage[utf8x]{inputenc}                               % Accept different input encodings
\usepackage{amsmath,amsfonts,amsthm,amssymb}               % Math packages to use equations
\usepackage{siunitx}                                       % Scientific units and numbering
\usepackage[usenames,dvipsnames,svgnames,table]{xcolor}    % Set color of text/background
\linespread{1.1}                                           % Default line spacing size
\usepackage{microtype}                                     % Improves spacing in the document
\usepackage{setspace}                                      % Set line spacing dynamically
\usepackage{tocloft}                                       % List adjustments including ToC
\usepackage{pgfplots, pgfplotstable}
%\usepackage{framed}
\usepackage[nocut]{thmbox}
\usepackage{footnote}
\usepackage{tablefootnote}
\usepackage{enumitem}

%----------------------------------------------------------------------------------------
%    FIGURES
%----------------------------------------------------------------------------------------

\usepackage{graphicx}                                      % Required for the inclusion of images
\graphicspath{{./Pictures/}}                               % Specifies picture directory
\usepackage{float}                                         % Allows putting an [H] in \begin{figure}
\usepackage{wrapfig}                                       % Allows in-line images
\usepackage{enumitem}
\usepackage[]{chapterbib}
\usepackage[titlenumbered,ruled]{algorithm2e}
\usepackage{verbatim}

\usepackage{hyperref}                                      % References
\usepackage{cleveref}                                      % Better References
%\crefname{lstlisting}{listing}{listings}
%\Crefname{lstlisting}{Listing}{Listings}
\crefname{figure}{figure}{figures}
\Crefname{figure}{Figure}{Figures}

%%% LINKS for ToC
\usepackage{hyperref}
\hypersetup{
   colorlinks,
   citecolor=black,
   filecolor=black,
   linkcolor=black,
   urlcolor=black
}

% make table at top even on an empty page
\makeatletter
    \setlength\@fptop{0\p@}
\makeatother
% to make nice multi column table
\usepackage{tabularx,booktabs}
\newcolumntype{Y}{>{\centering\arraybackslash}X}
\newcolumntype{s}{>{\hsize=.5\hsize}X}

%----------------------------------------------------------------------------------------
%    INCLUDE CODE
%----------------------------------------------------------------------------------------

\usepackage{listings}                                      % Package so code looks pretty
\lstset{
language=C,                                                % Choose the language
basicstyle=\footnotesize,                                  % The size of the fonts used
numbers=left,                                              % Where to put the line-numbers
numberstyle=\footnotesize,                                 % The size of the line-numbers
stepnumber=1,                                              % The step line-numbers
numbersep=5pt,                                             % How far the line-numbers are from the code
backgroundcolor=\color{white},                             % Choose the background color
showspaces=false,                                          % Show spaces adding partiular underscores
showstringspaces=false,                                    % Underline spaces within strings
showtabs=false,                                            % Show tabs within strings adding particular underscores
frame=single,                                              % Adds a frame around the code
tabsize=2,                                                 % Sets default tabsize to 2 spaces
captionpos=b,                                              % Sets the caption-position to bottom
breaklines=true,                                           % Sets automatic line breaking
breakatwhitespace=false,                                   % Sets if automatic breaks should only happen at whitespace
escapeinside={\%*}{*)}                                     % If you want to add a comment within your code
}

%----------------------------------------------------------------------------------------
%    COMMANDS
%----------------------------------------------------------------------------------------

\setlength\parindent{0pt}                                  % Removes all indentation from paragraphs
\setlength{\parskip}{4.0ex plus 0.5ex minus 0.3ex}         % Spacing between paragraphs
\renewcommand*\thesection{\arabic{section}}                % Renew section numbers
\renewcommand{\labelenumi}{\alph{enumi}.}                  % Section ordered numbering
\let\oldvec\vec                                            % Save the old vector style
\renewcommand{\vec}[1]{\boldsymbol{\mathbf{#1}}}
\DeclareMathOperator{\Tr}{Tr}                        % Set vectors to look like vectors
\renewcommand{\sfdefault}{phv}                             % Change default font
\renewcommand{\familydefault}{\sfdefault}                  % Use default font everywhere
\newcommand{\E}{\mathbb{E}}
\newcommand{\probP}{\mathbb{P}}
\newcommand{\probQ}{\mathbb{Q}}
\newcommand*{\Ito}{It\^{o} }
\newenvironment{concept}[1]
    {\vspace{5mm}
    \begin{thmbox}[L]{\textbf{#1}}
    }
    { 
    \end{thmbox}
    }
\makeatletter
\newenvironment{algoalign}
  {\setlength{\mathindent}{0pt}
   \vspace{-3mm}
   \start@align\@ne\st@rredtrue\m@ne
  }
  {\endalign
  \vspace{-8mm}
  }
\makeatother

\newcommand\numberthis{\addtocounter{equation}{1}\tag{\theequation}}


\newcommand{\argmax}{\operatornamewithlimits{argmax}}
\title{\vspace{-5ex}FE5116DP Assignment 1\vspace{-7ex}}
\date{}
\begin{document}
\maketitle
\subsection*{Assignment 1.1}
Given an integer $n$ in decimal notation with no leading zeroes $(10 < n < 10^{10})$,
and denote $m$ the number of digits,
perform the following operation exactly $k$ times: \vspace{-0.5cm}
\begin{itemize}
\item Choose two different positions $i$ and $j$ such that $1 \leq i < j \leq m$. 
\item Swap the digits at positions i and j. 
\end{itemize}
\vspace{-0.5cm}
The $k$ operations should be chosen such that the resulting number is the maximal possible number you can get at the end of this procedure. 

Design and implement a function in Octave to solve the above problem:
\begin{verbatim}
function result = swapMax(n, k)
\end{verbatim}

\emph{Examples} \vspace{-0.9cm}
\begin{verbatim}
swapMax(456, 1) = 654
swapMax(9754, 1) = 9745
swapMax(435528, 2) = 855324
swapMax(3648935, 1) = 9648335
swapMax(3648935, 2) = 9846335
\end{verbatim}

\emph{Hint}

You might consider using recursion (it's not compulsory though):  \vspace{-0.9cm}
\begin{verbatim}
swapMax(n, k) = swapMax(swapMax(n, k-1), 1)
\end{verbatim}
The strategy of swapMax(n, 1) is simple: starting from the left most digit, 
the i-th position should be placed the i-th largest digit, 
if it is not the case, you should swap i and j where j is the position of the i-th largest digit.

Note that you have to perform k operations even if the maximum number is achieved before k operations.
In that case, the swap that is the least harmful should be used.


\subsection*{Assignment 1.2}
Implement two functions for approximating $e^x$:
\begin{itemize}
\item myExp1($x$, $n$): $\displaystyle e^x = \lim_{n\rightarrow\infty}\left(1+\frac x n \right)^n$
\item myExp2($x$, $n$): $\displaystyle e^x = \sum_{n=0}^{\infty} \frac{x^n}{n!} = 1 + x + \frac{x^2}{2!} + \frac{x^3}{3!} + \ldots$
\end{itemize}
Plot convergence charts for $x=1$ and $x=2$.
The X-axis of the chart is the number of terms used, i.e.\ $n$.
The Y-axis of the chart is the error versus the Octave built-in \verb=exp= function.
The Y-axis should be plot in log-scale.
For each $x$, the convergence charts of the two functions should be plotted together for the ease of comparison.

\subsection*{Assignment 1.3}

\paragraph{a)} Implied volatility is the volatility $\sigma$,
when plugged into the Black-Scholes option price formula, 
generates the option price $V$ quoted in the market.
The Black-Scholes analytic formula for a call option on the underlying $S$ struck at $K$ is
\begin{align}
V(S, r, K, \sigma, T) = S N(d_+) - K e^{-rT} N(d_-)
\label{lab:bs-call}
\end{align}
where $r$ is the risk-free interest rate,
$N(\cdot)$ is the standard cumulative normal function and is implemented as \verb=normcdf= in Octave, and
\begin{align*}
d_{\pm} = \frac{ln\frac{S}{K} + (r \pm \frac{1}{2}\sigma^2)T}{\sigma\sqrt{T}}.
\end{align*}
Implement a function 
\begin{verbatim}
  ImpliedVol(S, r, K, T, V)
\end{verbatim}
that calculates the volatility $\sigma$ from all other given parameters.
Note that \eqref{lab:bs-call} is an implicit function of $\sigma$ so a root search is needed to implied $\sigma$.

Calculate the implied volatility for the below inputs: \vspace{-0.5cm}
\begin{verbatim}
  S = 100, r = 0.02, T = 1, K = 102, V = 8
\end{verbatim}

\paragraph{b)} The Black-Scholes assumption does not hold in real market 
--- the volatility is not constant.
The implied volatility for European options expired at time $T$ 
is a function $\sigma(K)$ of the option strike $K$.
This is called the volatility smile.

Normally the market does not quote the whole volatility smile.
It only quotes a discrete set of instruments.
So the construction of the volatility smile is left to each individual financial institute 
with a choice of interpolation scheme, such that option of all strikes can be evaluated.

Let us consider a simplified situation. 
Assuming you have 5 market quotes $[(K_i, \sigma_i)]$ for $i\in\{1, 2, 3, 4, 5\}$,
construct a function 
\vspace{-0.6cm}
\begin{verbatim}
  VolSmile( ks, vs, strike )
    ks: the 5 strike quotes, in ascending order
    vs: the 5 implied volatility quotes corresponding to ks
    strike: the strike you want to query
\end{verbatim}
\vspace{-0.6cm}
using cubic spline interpolator provided by Octave (
\url{https://www.gnu.org/software/octave/doc/interpreter/One_002ddimensional-Interpolation.html})
for the strikes within the range of $[K_1, K_5]$. For the ranges outside, we extrapolate using $\sigma_1$ for the left wing and $\sigma_5$ for the right wing (flat wings).
The function returns a $\sigma$ for any positive input \verb=strike=.

Construct the volatility smile for the below inputs:
\begin{verbatim}
  ks = [78, 92, 102, 115, 130]
  vs = [0.30, 0.25, 0.20, 0.23, 0.28]
\end{verbatim}
and plot the smile for strike from [1,2,3,...,200].

\paragraph{c)}
With the whole volatility smile constructed from step b, we can 
calculate the prices of European call option using Black-Scholes formula \eqref{lab:bs-call} for any strike.
For strike from [1, 2,...,200] and
\verb_S = 100, r = 0.02, T = 1_,
plot the chart of option prices with respect to strike.

\paragraph{d)}
Step c gives us the function $C(K)$ from strike $K$ to price of European call option.
The second derivative of the function $C(K)$ with respect to $K$ is in fact linked to
the probability density function $p(.)$ of the spot price at expiry under risk-neutral measure.
That is:
\begin{align*}
\frac{\partial^2 C(K)}{\partial K^2} = e^{-rT} p(K).
\end{align*}

For the same set of inputs, use numerical derivative to calculate $p(K)$ for $K$ from [1, 2,...,200], and plot it.

\paragraph{e)}

The present value of a European call option with strike at $K$ and expiry at $T$ can be expressed as expectation under risk-neutral measure $\probQ$
\begin{align*}
V(S, K, T) & = e^{-rT}\E_\probQ [(S_T - K)_+]
\end{align*}
and is the integral
\begin{align}
\label{eqn:bssolution}
  V(S_0, K, T) = e^{-rT}\int_K^{+\infty} \left(S_T - K\right) p(S_T) dS_T 
\end{align}
Step d gives us a way to estimate the probability density function $p(.)$,
so we can use quadrature to integrate \eqref{eqn:bssolution}. 

Calculate call option prices using mid-point rule quadrature with 1000 intervals for strike from [1, 2, ..., 200],
and plot the difference of the prices from the result of step c.

\subsection*{Submission Checklist}
Below is the list of required files in the submission:
\begin{itemize}
\item \verb=doc.pdf= that documents the algorithms, implementations, the required answers and charts. Only PDF format is acceptable, please do not submit any MS Word file.
\item \verb=swapMax.m=  that contains the implementation of the required function for assignment 1.1.
\item \verb=ex2.m= that contains the functions required by assignment 1.2, and the test script that generates the charts and necessary information.
\item \verb=ex3.m= that contains the functions required by assignment 1.3, and the test script that generates the charts and necessary information.
\end{itemize}
Readability and performance of the code might be taken into account at grading.

You do not need to create sub-folders for the questions. 
The filename of the zip file should be in the format of \verb=Ex1_name1_matricNumber1_name2_matricNumber2.zip=.
And the matric number should contain the last alphabetic letter.

\end{document}
