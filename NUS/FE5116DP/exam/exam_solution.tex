
\documentclass[11pt,a4paper,hidelinks,fleqn]{article}            % Article 12pt font for a4 paper while hiding links
\usepackage[margin=1in]{geometry}                          % Required to adjust margins

%----------------------------------------------------------------------------------------
%    TYPE SETTING PACKAGES
%----------------------------------------------------------------------------------------

\usepackage[english]{babel}                                % English language/hyphenation 
\usepackage[utf8x]{inputenc}                               % Accept different input encodings
\usepackage{amsmath,amsfonts,amsthm,amssymb}               % Math packages to use equations
\usepackage{siunitx}                                       % Scientific units and numbering
\usepackage[usenames,dvipsnames,svgnames,table]{xcolor}    % Set color of text/background
\linespread{1.1}                                           % Default line spacing size
\usepackage{microtype}                                     % Improves spacing in the document
\usepackage{setspace}                                      % Set line spacing dynamically
\usepackage{tocloft}                                       % List adjustments including ToC
\usepackage{pgfplots, pgfplotstable}
%\usepackage{framed}
\usepackage[nocut]{thmbox}
\usepackage{footnote}
\usepackage{tablefootnote}
\usepackage{enumitem}

%----------------------------------------------------------------------------------------
%    FIGURES
%----------------------------------------------------------------------------------------

\usepackage{graphicx}                                      % Required for the inclusion of images
\graphicspath{{./Pictures/}}                               % Specifies picture directory
\usepackage{float}                                         % Allows putting an [H] in \begin{figure}
\usepackage{wrapfig}                                       % Allows in-line images
\usepackage{enumitem}
\usepackage[]{chapterbib}
\usepackage[titlenumbered,ruled]{algorithm2e}
\usepackage{verbatim}

\usepackage{hyperref}                                      % References
\usepackage{cleveref}                                      % Better References
%\crefname{lstlisting}{listing}{listings}
%\Crefname{lstlisting}{Listing}{Listings}
\crefname{figure}{figure}{figures}
\Crefname{figure}{Figure}{Figures}

%%% LINKS for ToC
\usepackage{hyperref}
\hypersetup{
   colorlinks,
   citecolor=black,
   filecolor=black,
   linkcolor=black,
   urlcolor=black
}

% make table at top even on an empty page
\makeatletter
    \setlength\@fptop{0\p@}
\makeatother
% to make nice multi column table
\usepackage{tabularx,booktabs}
\newcolumntype{Y}{>{\centering\arraybackslash}X}
\newcolumntype{s}{>{\hsize=.5\hsize}X}

%----------------------------------------------------------------------------------------
%    INCLUDE CODE
%----------------------------------------------------------------------------------------

\usepackage{listings}                                      % Package so code looks pretty
\lstset{
language=C,                                                % Choose the language
basicstyle=\footnotesize,                                  % The size of the fonts used
numbers=left,                                              % Where to put the line-numbers
numberstyle=\footnotesize,                                 % The size of the line-numbers
stepnumber=1,                                              % The step line-numbers
numbersep=5pt,                                             % How far the line-numbers are from the code
backgroundcolor=\color{white},                             % Choose the background color
showspaces=false,                                          % Show spaces adding partiular underscores
showstringspaces=false,                                    % Underline spaces within strings
showtabs=false,                                            % Show tabs within strings adding particular underscores
frame=single,                                              % Adds a frame around the code
tabsize=2,                                                 % Sets default tabsize to 2 spaces
captionpos=b,                                              % Sets the caption-position to bottom
breaklines=true,                                           % Sets automatic line breaking
breakatwhitespace=false,                                   % Sets if automatic breaks should only happen at whitespace
escapeinside={\%*}{*)}                                     % If you want to add a comment within your code
}

%----------------------------------------------------------------------------------------
%    COMMANDS
%----------------------------------------------------------------------------------------

\setlength\parindent{0pt}                                  % Removes all indentation from paragraphs
\setlength{\parskip}{4.0ex plus 0.5ex minus 0.3ex}         % Spacing between paragraphs
\renewcommand*\thesection{\arabic{section}}                % Renew section numbers
\renewcommand{\labelenumi}{\alph{enumi}.}                  % Section ordered numbering
\let\oldvec\vec                                            % Save the old vector style
\renewcommand{\vec}[1]{\boldsymbol{\mathbf{#1}}}
\DeclareMathOperator{\Tr}{Tr}                        % Set vectors to look like vectors
\renewcommand{\sfdefault}{phv}                             % Change default font
\renewcommand{\familydefault}{\sfdefault}                  % Use default font everywhere
\newcommand{\E}{\mathbb{E}}
\newcommand{\probP}{\mathbb{P}}
\newcommand{\probQ}{\mathbb{Q}}
\newcommand*{\Ito}{It\^{o} }
\newenvironment{concept}[1]
    {\vspace{5mm}
    \begin{thmbox}[L]{\textbf{#1}}
    }
    { 
    \end{thmbox}
    }
\makeatletter
\newenvironment{algoalign}
  {\setlength{\mathindent}{0pt}
   \vspace{-3mm}
   \start@align\@ne\st@rredtrue\m@ne
  }
  {\endalign
  \vspace{-8mm}
  }
\makeatother

\newcommand\numberthis{\addtocounter{equation}{1}\tag{\theequation}}


\newcommand{\argmax}{\operatornamewithlimits{argmax}}
\date{}
\begin{document}

\subsection*{Question 1 [20 marks]}

\paragraph{a)} For the function $f(x) = x^4 + 2 x^2 - 5x, ~x\in[0, 10]$,
describe an algorithm for finding the root of $f'(x) = 0$ using the secant method.

Solution:
\begin{align*}
g(x) & = f'(x) = 4x^3 + 4 x - 5 \\
x_0 & = 0 \\
x_1 & = 10 \\
x_{i+1} & = x_i - g(x_i) [x_i - x_{i-1}] / [g(x_i) - g(x_{i-1})]
\end{align*}
until $x_{i+1} - x_{i} < \text{tolerance}$.

\paragraph{b)} Calculate the first two iterations of the secant method used in a).

Solution:
\begin{align*}
g(x_0) & = -5, ~g(x_1) = 4035, \\
x_2 & = 10 - 4035 \times 10 / 4040 = 0.0124, g(x_2) = -4.95, \\
x_3 & = 0.0124 + 4.95 \times (-9.9876) / (-4.95 - 4035) = 0.0224
\end{align*}


\subsection*{Question 2 [20 marks]} 

The function $f(x)$ is four times continuously differentiable ($C^4$).
Given 5 sample points of the function $[f(x_0-h), f(x_0-\frac{h}{2}), f(x_0), f(x_0+\frac{h}{2}), f(x_0+h)]$,
describe the algorithm to approximate $f'(x_0)$ through:

\vspace{-6mm}
\paragraph{a)} Taylor expansion at the four points $f(x_0-h), f(x_0-\frac{h}{2}), f(x_0+\frac{h}{2}), f(x_0+h)$;

Solution:
The Taylor expansion multiplied by $b_i$ is
\begin{align*}
\begin{cases}
b_1 f(x - h) & = b_1 [f(x) - f'(x)h + \frac12 f''(x)h^2 - \frac16 f^{(3)}(x)h^3] \\
b_2 f(x - \frac12 h) & = b_2 [f(x) - f'(x) \frac12 h + \frac12 f''(x) \frac14 h^2 - \frac16 f^{(3)}(x)\frac18 h^3] \\
b_3 f(x + \frac12 h) & = b_3 [f(x) + f'(x) \frac12 h + \frac12 f''(x) \frac14 h^2 + \frac16 f^{(3)}(x) \frac18 h^3] \\
b_4 f(x + h) & = b_4 [f(x) + f'(x)h + \frac12 f''(x)h^2 + \frac16 f^{(3)}(x)h^3]
\end{cases}
\end{align*}
We would like the sum of the right hand sides to be equal to $f'(x)$, 
so the $b_i$'s have to satisfy the linear system:
\begin{align*}
\begin{cases}
& -b_1 - \frac12 b_2   + \frac12 b_3 + b_4 = \frac1h \\
& b_1  + \frac14 b_2  + \frac14 b_3 + b_4 = 0   \\
& b_1  + b_2      + b_3     + b_4 = 0   \\
& -b_1 - \frac18 b_2  + \frac18 b_3 + b_4 = 0   
\end{cases} 
\Rightarrow  \vec{M} \vec{b} = \left[\frac1h\ 0\ 0\ 0 \right]^{\top}
\Rightarrow  \vec{b} = \vec{M}^{-1}\left[\frac1h\ 0\ 0\ 0 \right]^{\top}
\end{align*}
And sum of the left hand side is $\sum_i cs_i y_i = f'(x)$, where $cs = [b_1, b_2, 0, b_3, b_4] $.


\paragraph{b)} Richardson extrapolation with the central difference g(h) and the step sizes $h$ and $\displaystyle \frac{h}{2}$; 

Solution:

The central differences are
\begin{align*}
& g(h) = \frac{f(x+h) - f(x-h)}{2h} \\
& g\left(\frac{h}{2}\right) = \frac{f(x+\frac h 2) - f(x-\frac h 2)}{h}
\end{align*}
Richardson extrapolation:
\begin{align*}
f'(x) = \frac43 g\left(\frac{h}{2}\right) - \frac13 g(h) = \frac1{6h} f(x-h) - \frac{4}{3h} f(x-\frac{h}2) + \frac{4}{3h} f(x+\frac{h}2) - \frac1{6h} f(x+h)
\end{align*}


\paragraph{c)} constructing the polynomial of order 4, $p(x) = \sum_{i=0}^4 a_i x^i$, passing the 5 given points and taking the derivative of the polynomial at $x_0$.

Solution:

Given five points we can fit a 4th order polynomial 
\begin{align*}
y = a_4 x^4 + a_3 x^3 + a_2 x^2 + a_1 x + a_0
\end{align*}
by solving the linear system $\vec{M} \vec{a} = \vec{y}$, where
\begin{verbatim}
M  = [(x0-h)^4, (x0-h)^3, (x0-h)^2, (x0-h), 1;
      (x0-h/2)^4, (x0-h/2)^3, (x0-h/2)^2, (x0-h/2), 1;
      x0^4, 	x0^3, 	  x0^2,     x0,     1;
      (x0+h/2)^4, (x0+h/2)^3, (x0+h/2)^2, (x0+h/2), 1;
      (x0+h)^4, (x0+h)^3, (x0+h)^2, (x0+h), 1;]
\end{verbatim}
So the coefficients of the polynomial $\vec{a} = \vec{M}^{-1} \vec{y}$.
And 
\begin{align*}
f'(x) = [4x^3, 3x^2, 2x, 1, 0]^\top \vec{a}
\end{align*}



\subsection*{Question 3 [20 marks]}
Consider the following one step binomial tree model of the stock price process:
\begin{figure*}[h]
\includegraphics[scale=0.9]{./4}
\end{figure*}

a) Assume interest rate is $0\%$, compute the price of a call option struck at 100 at maturity 1 year?

Solution: risk neutral probability: 50\%, 50\%. option price $= 10 \times 0.5 + 0 * 0.5 = 5$.

b) Assume interest rate is $5\%$, compute the price of a call option struck at 100 at maturity 1 year?

Solution: risk neutral probability: 75\%, 25\%. option price $= 10 \times 0.75 + 0 * 0.5 = 7.5$.


\subsection*{Question 4 [20 marks]} 
Consider the numerical integration of 
\begin{align*}
\int_0^{2} f(x) dx, ~~\text{where~} f(x) = \frac{1}{2} e^{-\frac12x^2}
\end{align*}

\paragraph{a)} Evaluate the integral by mid-point quadrature with 4 equally spaced intervals. 

Solution:
Intervals are (0, 2.5), (2.5, 5), (5, 7.5), (7.5, 10)
\begin{align*}
f(0.25) & = 0.48462, 
~f(0.75) = 0.37742, 
~f(1.25) = 0.22892, 
~f(0.875) = 0.10813  \\
ans = & 2.5 [f(1.25) + f(3.75) + f(6.25) + f(8.75)] = 0.11468
\end{align*}


\paragraph{b)} Describe in pseudo code the algorithm to evaluate the integral using Monte-Carlo simulation.

Solution:

For i = 1: N
\begin{enumerate}
\item[]{1.} Draw random number x' from uniform distribution [0, 1].
\item[]{2.} $x=2x', ~~r_i = f(x)$,
\end{enumerate}
result is $\frac1N \sum(r_i)$
 
\paragraph{c)} You would like to use $g(x)$, the second order Taylor 
expansion of $f(x)$ at $x=0$, as control variate. 
Assume you know already the variance of $g(x)$ and covariance of $[f, g]$.
Write down the expression of $g(x)$, 
and explain what change you need to make in the Monte-Carlo integration algorithm in b) to reduce the Monte-Carlo noise.

Solution:
\begin{align*}
f'(x) = -\frac12 x e^{-\frac12 x^2}, ~~f''(x) = -\frac12 (e^{-\frac12 x^2} - x^2 e^{-\frac12 x^2})
\end{align*}
Taylor expansion at $x=0$ is 
\begin{align*}
g(x) = f(0) + f'(0) x + \frac12 f''(0) x^2 = \frac12 - \frac14 x^2 \\
E[g(x)] = \int_0^2 g(x) dx = 1 - 2/3 = 0.333
\end{align*}

Instead of $r_i = f(x)$ in step 2 of b), 
we use $r_i = f(x) - \lambda(g(x) - 0.333)$, where $\lambda = \frac{covar(f, g)}{var(g)}$.

\subsection*{Question 5 [20 marks]}
Consider the process $X_t$
\begin{align*}
dX_t = \sigma dW_t
\end{align*}
where $W_t$ is a standard Wiener process and 
$\sigma$ is the constant volatility.
Denote $V_t$ the price of a derivative on $X_t$.

\paragraph{a)} Derive the partial differential equation satisfied by $V(x, t)$.

Solution:

Assuming the interest rate is constant $r$,
the price of the discounted derivative contract $V(t, Z(t))$ follows the SDE:
\begin{align}
d\left(\frac{V}{B}\right) & = \frac{1}{B}\left(\frac{\partial V}{\partial t} dt + \frac{\partial V}{\partial X} dX + \frac{1}{2}\frac{\partial^2 V}{\partial X^2} dX^2 \right) - r\frac{V}{B}dt  \\
   & = \frac{1}{B} \left[\left(\frac{\partial V}{\partial t} + \frac{1}{2} \sigma^2 \frac{\partial^2 X}{\partial X^2} - rV\right)dt + \sigma \frac{\partial V}{\partial X} dW_t \right]
\end{align}
For $\frac{V}{B}$ to be a martingale, 
the derivative contract price $V(t, Z(t))$ has to satisfy the PDE:
\begin{align}
rV(t) = \frac{\partial V(t)}{\partial t} + \frac{1}{2} \sigma^2 \frac{\partial^2 V(t)}{\partial Z^2}
\end{align}


\paragraph{b)} Assume the risk-free interest rate is 0,
discretize the partial differential equation using finite difference in both time and x axis with explicit Euler scheme,
write down the equation that computes $V(x, t_i)$ from the time slice $t_{i+1}$.

Solution: temporal time step h, spatial time step b,
\begin{align}
V(x, t_{i}) = V(x, t_{i+1}) +  \frac{1}{2} \sigma^2 h \frac{V(x-b, t_{i+1}) + V(x+b, t_{i+1}) - 2 V(x, t_{i+1})}{b^2}
\end{align}


c) Suppose that the derivative's price at $t_{i+1}$ is of the form shown in the below chart,
i.e., $V(x, t_{i+1})$ is known, based on the result of b),
sketch the rough shape of $V(x, t_{i})$ on top of the plot and explain the rationale.

Solution:

At $t_i$, the curve is higher than $V(t_{i+1})$ when the second derivative is positive, and lower when second derivative is negative.

\begin{figure*}[h]
\includegraphics[scale=0.9]{./6s} \\
\vspace{1cm}
\textbf{*END OF PAPER*}
\end{figure*}



\end{document}
