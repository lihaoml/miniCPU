
\documentclass[11pt,a4paper,hidelinks,fleqn]{article}            % Article 12pt font for a4 paper while hiding links
\usepackage[margin=1in]{geometry}                          % Required to adjust margins

%----------------------------------------------------------------------------------------
%    TYPE SETTING PACKAGES
%----------------------------------------------------------------------------------------

\usepackage[english]{babel}                                % English language/hyphenation 
\usepackage[utf8x]{inputenc}                               % Accept different input encodings
\usepackage{amsmath,amsfonts,amsthm,amssymb}               % Math packages to use equations
\usepackage{siunitx}                                       % Scientific units and numbering
\usepackage[usenames,dvipsnames,svgnames,table]{xcolor}    % Set color of text/background
\linespread{1.1}                                           % Default line spacing size
\usepackage{microtype}                                     % Improves spacing in the document
\usepackage{setspace}                                      % Set line spacing dynamically
\usepackage{tocloft}                                       % List adjustments including ToC
\usepackage{pgfplots, pgfplotstable}
%\usepackage{framed}
\usepackage[nocut]{thmbox}
\usepackage{footnote}
\usepackage{tablefootnote}
\usepackage{enumitem}

%----------------------------------------------------------------------------------------
%    FIGURES
%----------------------------------------------------------------------------------------

\usepackage{graphicx}                                      % Required for the inclusion of images
\graphicspath{{./Pictures/}}                               % Specifies picture directory
\usepackage{float}                                         % Allows putting an [H] in \begin{figure}
\usepackage{wrapfig}                                       % Allows in-line images
\usepackage{enumitem}
\usepackage[]{chapterbib}
\usepackage[titlenumbered,ruled]{algorithm2e}
\usepackage{verbatim}

\usepackage{hyperref}                                      % References
\usepackage{cleveref}                                      % Better References
%\crefname{lstlisting}{listing}{listings}
%\Crefname{lstlisting}{Listing}{Listings}
\crefname{figure}{figure}{figures}
\Crefname{figure}{Figure}{Figures}

%%% LINKS for ToC
\usepackage{hyperref}
\hypersetup{
   colorlinks,
   citecolor=black,
   filecolor=black,
   linkcolor=black,
   urlcolor=black
}

% make table at top even on an empty page
\makeatletter
    \setlength\@fptop{0\p@}
\makeatother
% to make nice multi column table
\usepackage{tabularx,booktabs}
\newcolumntype{Y}{>{\centering\arraybackslash}X}
\newcolumntype{s}{>{\hsize=.5\hsize}X}

%----------------------------------------------------------------------------------------
%    INCLUDE CODE
%----------------------------------------------------------------------------------------

\usepackage{listings}                                      % Package so code looks pretty
\lstset{
language=C,                                                % Choose the language
basicstyle=\footnotesize,                                  % The size of the fonts used
numbers=left,                                              % Where to put the line-numbers
numberstyle=\footnotesize,                                 % The size of the line-numbers
stepnumber=1,                                              % The step line-numbers
numbersep=5pt,                                             % How far the line-numbers are from the code
backgroundcolor=\color{white},                             % Choose the background color
showspaces=false,                                          % Show spaces adding partiular underscores
showstringspaces=false,                                    % Underline spaces within strings
showtabs=false,                                            % Show tabs within strings adding particular underscores
frame=single,                                              % Adds a frame around the code
tabsize=2,                                                 % Sets default tabsize to 2 spaces
captionpos=b,                                              % Sets the caption-position to bottom
breaklines=true,                                           % Sets automatic line breaking
breakatwhitespace=false,                                   % Sets if automatic breaks should only happen at whitespace
escapeinside={\%*}{*)}                                     % If you want to add a comment within your code
}

%----------------------------------------------------------------------------------------
%    COMMANDS
%----------------------------------------------------------------------------------------

\setlength\parindent{0pt}                                  % Removes all indentation from paragraphs
\setlength{\parskip}{4.0ex plus 0.5ex minus 0.3ex}         % Spacing between paragraphs
\renewcommand*\thesection{\arabic{section}}                % Renew section numbers
\renewcommand{\labelenumi}{\alph{enumi}.}                  % Section ordered numbering
\let\oldvec\vec                                            % Save the old vector style
\renewcommand{\vec}[1]{\boldsymbol{\mathbf{#1}}}
\DeclareMathOperator{\Tr}{Tr}                        % Set vectors to look like vectors
\renewcommand{\sfdefault}{phv}                             % Change default font
\renewcommand{\familydefault}{\sfdefault}                  % Use default font everywhere
\newcommand{\E}{\mathbb{E}}
\newcommand{\probP}{\mathbb{P}}
\newcommand{\probQ}{\mathbb{Q}}
\newcommand*{\Ito}{It\^{o} }
\newenvironment{concept}[1]
    {\vspace{5mm}
    \begin{thmbox}[L]{\textbf{#1}}
    }
    { 
    \end{thmbox}
    }
\makeatletter
\newenvironment{algoalign}
  {\setlength{\mathindent}{0pt}
   \vspace{-3mm}
   \start@align\@ne\st@rredtrue\m@ne
  }
  {\endalign
  \vspace{-8mm}
  }
\makeatother

\newcommand\numberthis{\addtocounter{equation}{1}\tag{\theequation}}


\newcommand{\argmax}{\operatornamewithlimits{argmax}}
\date{}
\begin{document}
\subsection*{Question 1 [25 marks]} 
Write a function \verb=MyFirstDeriv(f, x0, h)= to compare three methods of computing the derivative $f'(x_0)$
at point $x_0$ for a function $f(x)$ of an unknown form.
The input \verb=x0= and \verb=h= are two double variables,
and \verb=f= is a function from double to double. 

The first step of \verb=MyFirstDeriv(f, x0, h)= is to construct 5 sample points: $[f(x_0-h), f(x_0-\frac{h}{2}), f(x_0), f(x_0+\frac{h}{2}), f(x_0+h)]$, denoted as \verb=ys=.
Then, it derives the first derivative $f'(x_0)$ in three ways:

\vspace{-6mm}
\paragraph{a)} List the four taylor expansion at the four points $f(x_0-h), f(x_0-\frac{h}{2}), f(x_0+\frac{h}{2}), f(x_0+h)$, 
multiply each by coefficient $b_j$ for $j\in\{1, 2, 3, 4\}$, 
construct a linear system of $b_j$ such that the order 0, 2, and 3 disappear, and order 1 term becomes 1, 
solve for $b_i$, 
print out $b_i$,
derive the coefficients \verb=cs= such that $f'(x_0)$ = \verb=sum(cs .* ys)=,
and print out \verb=cs= and $f'(x_0)$. 
\vspace{-6mm}
\paragraph{b)} Compute $f'(x_0)$ using Richardson extrapolation with the central difference g(h) and the step sizes $h$ and $\displaystyle \frac{h}{2}$,
derived the coefficients \verb=cs= such that $f'(x_0)$ = \verb=sum(cs .* ys)=,
print out \verb=cs= and $f'(x_0)$. Compare the coefficients \verb=cs= with the coefficients derived in part \textbf{a)}.
\vspace{-6mm}
\paragraph{c)} Construct a $4^{th}$ order polynomial $p(x) = \sum_{i=0}^4 a_i x^i$ that passes through the five given points, 
compute $p'(x_0)$ analytically, 
print out the polynomial,
print out the coefficients \verb=cs= such that $p'(x_0)$ = \verb=sum(cs .* ys)=,
and print out the derivative $p'(x_0)$.
Again, compare the coefficients \verb=cs= with the coefficients derived from in \textbf{a)}.


\subsection*{Question 2 [15 marks]}
Numerical minimization of the function
\begin{align*}
f(x) = x^4 - 20 x^2 + 1
\end{align*}
can be viewed as finding roots of the first derivative of f(x),

a) Describe an algorithm for doing this using the secant method.

b) Calculate the first two steps of the algorithm.


\subsection*{Question 3 [20 marks]}
Consider the following one step binomial tree model of the stock price process:
\begin{figure*}[h]
\includegraphics[scale=0.9]{./4}
\end{figure*}

a) Assume interest rate is $0\%$, compute the price of a call option struck at 100 at maturity 1 year?

a) Assume interest rate is $5\%$, compute the price of a call option struck at 100 at maturity 1 year?


\subsection*{Question 4 [20 marks]} 



\subsection*{Question 5 [20 marks]}
Consider the process of a underlying asset $X_t$
\begin{align*}
dX_t = \sigma dW_t
\end{align*}
where $W_t$ is a standard Wiener process and $\sigma$ is the constant volatility.
Denote $V_t$ the price of a derivative on $X_t$. 

a) Derive the partial differential equation satisfied by $V(x, t)$.

b) Discretize the partial differential equation using explicit Euler scheme,
write down the equation that computes $V(x, t_i)$ from the time slice $t_{i+1}$.

c) Suppose that the derivative's price at $t_{i+1}$ is of the following form,
sketch the rough shape of $V(x, t_{i+1})$ on top of the given plot and explain the rationale.
\begin{figure*}[h]
\includegraphics[scale=0.9]{./6c}
\end{figure*}

\textbf{*END OF PAPER*}
\end{document}
