
\documentclass[11pt,a4paper,hidelinks,fleqn]{article}            % Article 12pt font for a4 paper while hiding links
\usepackage[margin=1in]{geometry}                          % Required to adjust margins

%----------------------------------------------------------------------------------------
%    TYPE SETTING PACKAGES
%----------------------------------------------------------------------------------------

\usepackage[english]{babel}                                % English language/hyphenation 
\usepackage[utf8x]{inputenc}                               % Accept different input encodings
\usepackage{amsmath,amsfonts,amsthm,amssymb}               % Math packages to use equations
\usepackage{siunitx}                                       % Scientific units and numbering
\usepackage[usenames,dvipsnames,svgnames,table]{xcolor}    % Set color of text/background
\linespread{1.1}                                           % Default line spacing size
\usepackage{microtype}                                     % Improves spacing in the document
\usepackage{setspace}                                      % Set line spacing dynamically
\usepackage{tocloft}                                       % List adjustments including ToC
\usepackage{pgfplots, pgfplotstable}
%\usepackage{framed}
\usepackage[nocut]{thmbox}
\usepackage{footnote}
\usepackage{tablefootnote}
\usepackage{enumitem}

%----------------------------------------------------------------------------------------
%    FIGURES
%----------------------------------------------------------------------------------------

\usepackage{graphicx}                                      % Required for the inclusion of images
\graphicspath{{./Pictures/}}                               % Specifies picture directory
\usepackage{float}                                         % Allows putting an [H] in \begin{figure}
\usepackage{wrapfig}                                       % Allows in-line images
\usepackage{enumitem}
\usepackage[]{chapterbib}
\usepackage[titlenumbered,ruled]{algorithm2e}
\usepackage{verbatim}

\usepackage{hyperref}                                      % References
\usepackage{cleveref}                                      % Better References
%\crefname{lstlisting}{listing}{listings}
%\Crefname{lstlisting}{Listing}{Listings}
\crefname{figure}{figure}{figures}
\Crefname{figure}{Figure}{Figures}

%%% LINKS for ToC
\usepackage{hyperref}
\hypersetup{
   colorlinks,
   citecolor=black,
   filecolor=black,
   linkcolor=black,
   urlcolor=black
}

% make table at top even on an empty page
\makeatletter
    \setlength\@fptop{0\p@}
\makeatother
% to make nice multi column table
\usepackage{tabularx,booktabs}
\newcolumntype{Y}{>{\centering\arraybackslash}X}
\newcolumntype{s}{>{\hsize=.5\hsize}X}

%----------------------------------------------------------------------------------------
%    INCLUDE CODE
%----------------------------------------------------------------------------------------

\usepackage{listings}                                      % Package so code looks pretty
\lstset{
language=C,                                                % Choose the language
basicstyle=\footnotesize,                                  % The size of the fonts used
numbers=left,                                              % Where to put the line-numbers
numberstyle=\footnotesize,                                 % The size of the line-numbers
stepnumber=1,                                              % The step line-numbers
numbersep=5pt,                                             % How far the line-numbers are from the code
backgroundcolor=\color{white},                             % Choose the background color
showspaces=false,                                          % Show spaces adding partiular underscores
showstringspaces=false,                                    % Underline spaces within strings
showtabs=false,                                            % Show tabs within strings adding particular underscores
frame=single,                                              % Adds a frame around the code
tabsize=2,                                                 % Sets default tabsize to 2 spaces
captionpos=b,                                              % Sets the caption-position to bottom
breaklines=true,                                           % Sets automatic line breaking
breakatwhitespace=false,                                   % Sets if automatic breaks should only happen at whitespace
escapeinside={\%*}{*)}                                     % If you want to add a comment within your code
}

%----------------------------------------------------------------------------------------
%    COMMANDS
%----------------------------------------------------------------------------------------

\setlength\parindent{0pt}                                  % Removes all indentation from paragraphs
\setlength{\parskip}{4.0ex plus 0.5ex minus 0.3ex}         % Spacing between paragraphs
\renewcommand*\thesection{\arabic{section}}                % Renew section numbers
\renewcommand{\labelenumi}{\alph{enumi}.}                  % Section ordered numbering
\let\oldvec\vec                                            % Save the old vector style
\renewcommand{\vec}[1]{\boldsymbol{\mathbf{#1}}}
\DeclareMathOperator{\Tr}{Tr}                        % Set vectors to look like vectors
\renewcommand{\sfdefault}{phv}                             % Change default font
\renewcommand{\familydefault}{\sfdefault}                  % Use default font everywhere
\newcommand{\E}{\mathbb{E}}
\newcommand{\probP}{\mathbb{P}}
\newcommand{\probQ}{\mathbb{Q}}
\newcommand*{\Ito}{It\^{o} }
\newenvironment{concept}[1]
    {\vspace{5mm}
    \begin{thmbox}[L]{\textbf{#1}}
    }
    { 
    \end{thmbox}
    }
\makeatletter
\newenvironment{algoalign}
  {\setlength{\mathindent}{0pt}
   \vspace{-3mm}
   \start@align\@ne\st@rredtrue\m@ne
  }
  {\endalign
  \vspace{-8mm}
  }
\makeatother

\newcommand\numberthis{\addtocounter{equation}{1}\tag{\theequation}}


\newcommand{\argmax}{\operatornamewithlimits{argmax}}
\date{}
\begin{document}

\subsection*{Question 1 [20 marks]}

\paragraph{a)} For the function $f(x) = x^4 - 20 x^2 + 1, ~x\in[0, 10]$,
describe an algorithm for finding the root of $f'(x) = 0$ using the secant method.

\paragraph{b)} Calculate the first two iterations of the secant method used in a).

\subsection*{Question 2 [20 marks]} 

The function $f(x)$ is four times continuously differentiable ($C^4$).
Given 5 sample points of the function $[f(x_0-h), f(x_0-\frac{h}{2}), f(x_0), f(x_0+\frac{h}{2}), f(x_0+h)]$,
describe the algorithm to approximate $f'(x_0)$ through:

\paragraph{a)} Taylor expansion at the four points $f(x_0-h), f(x_0-\frac{h}{2}), f(x_0+\frac{h}{2}), f(x_0+h)$;

\paragraph{b)} Richardson extrapolation with the central difference g(h) and the step sizes $h$ and $\displaystyle \frac{h}{2}$; 

\paragraph{c)} constructing the polynomial of order 4, $p(x) = \sum_{i=0}^4 a_i x^i$, passing the 5 given points and taking the derivative of the polynomial at $x_0$.



\subsection*{Question 3 [20 marks]}
Consider the following one step binomial tree model of the stock price process:
\begin{figure*}[h]
\includegraphics[scale=0.9]{./4}
\end{figure*}

a) Assume interest rate is $0\%$, compute the price of a call option struck at 100 at maturity 1 year?

b) Assume interest rate is $5\%$, compute the price of a call option struck at 100 at maturity 1 year?


\subsection*{Question 4 [20 marks]} 
Consider the numerical integration of 
\begin{align*}
\int_0^{2} f(x) dx, ~~\text{where~} f(x) = \frac{1}{2} e^{-\frac12x^2}
\end{align*}

\paragraph{a)} Evaluate the integral by mid-point quadrature with 4 equally spaced intervals. 

\paragraph{b)} Describe in pseudo code the algorithm to evaluate the integral using Monte-Carlo simulation.
 

\paragraph{c)} You would like to use $g(x)$, the second order Taylor 
expansion of $f(x)$ at $x=0$, as control variate. 
Assume you know already the variance of $g(x)$ and covariance of $[f, g]$.
Write down the expression of $g(x)$, 
and explain what change you need to make in the Monte-Carlo integration algorithm in b) to use $g(x)$ as control variate.


\subsection*{Question 5 [20 marks]}
Consider the process $X_t$
\begin{align*}
dX_t = \sigma dW_t
\end{align*}
where $W_t$ is a standard Wiener process and $\sigma$ is the constant volatility.
Denote $V_t$ the price of a derivative on the process.

\paragraph{a)} Derive the partial differential equation satisfied by $V(x, t)$.

\paragraph{b)} Assume the risk-free interest rate is 0,
discretize the partial differential equation using finite difference in both time and x axis with explicit Euler scheme,
write down the equation that computes $V(x, t_i)$ from the time slice $t_{i+1}$.

\paragraph{c)} Suppose that the derivative's price at $t_{i+1}$ is of the form shown in the below chart,
i.e., $V(x, t_{i+1})$ is known, 
sketch the rough shape of $V(x, t_{i})$ on top of the plot and explain the rationale.
\begin{figure*}[h]
\includegraphics[scale=0.9]{./6c} \\
\vspace{1cm}
\textbf{*END OF PAPER*}
\end{figure*}



\end{document}
