
\documentclass[11pt,a4paper,hidelinks,fleqn]{article}            % Article 12pt font for a4 paper while hiding links
\usepackage[margin=1in]{geometry}                          % Required to adjust margins
../styleAndCommands.tex
\title{\vspace{-5ex}Exam, FE5116 (2014/2015, Semester 2)\vspace{-7ex}}
\date{}
\begin{document}
\maketitle
\subsection*{Question 1 [15 marks]} 
We want to model the process of stock price using the following SDE:
\begin{align*}
\frac{dS(t)}{S(t)} = \mu(t) dt + \sigma(t) dW_t
\end{align*}
and would like to use $X(t) = \ln(S(t))$ as the state variable.

a) What is the SDE for $X(t)$? [8 marks]
\begin{align*}
dX_t = (\mu(t) - \frac12 \sigma(t)^2) dt + \sigma(t) dW_t
\end{align*}

b) Given a standard normal random variable $z \sim N(0, 1)$ 
and time step $\Delta t$,
write down the equation that evolves $X$ from 
the $i$-th step to $i+1$ using Euler scheme, i.e., 
the function $f$ that $X_{i+1} = f(X_i, \Delta(t), z)$. [7 marks]
\begin{align*}
X_{i+1} = X_{i+1} + (\mu(t) - \frac12 \sigma(t)^2) \Delta t + \sigma(t) \sqrt{\Delta t} z
\end{align*}

\subsection*{Question 2 [15 marks]} 
Compute the integral $\int_0^2 (-x^2+2x+5) dx$ using the following quadrature with 2 intervals.

a) Mid-point rule [5 marks]
let $f = -x^2+2x+5$
\begin{align*}
f(0.5) \times 1 + f(1.5)\times 1 = 5.75 + 5.75 = 11.5
\end{align*}

b) Trapezoid rule [5 marks]
\begin{align*}
\frac{f(0) + f(1)}2 + \frac{f(1) + f(2)}2 = 5.5 + 5.5 = 11
\end{align*}

c) Simpson rule [5 marks]
\begin{align*}
(f(0) + 4f(1) + f(2)) \frac{1}{3} = (5 + 24 + 5) / 3 = \frac{34}3
\end{align*}

\subsection*{Question 3 [15 marks]} 
a) Set up a Newton's method to find $x_m$ and $m$ such that 
the line $y=mx$ is tangential to the curve $y = \sin(x)$ at $x_m$. [10 marks]
\begin{align*}
\sin(x) = mx \\
\cos(x) = m
\end{align*}
So the problem is to find the $x$ such that
\begin{align*}
f(x) = \sin(x) - x\cos(x) = 0
\end{align*}
Given $f'(x) = \cos(x) + x\sin(x) - \cos(x) = x \sin(x)$.
The Newton's method is
\begin{align*}
x_{i+1} = x_i - f(x_i) / f'(x_i)
\end{align*}

b) Start with $x_0 = \frac{\pi}{2}$, what is $x_1$ after one iteration of the Newton's method? [5 marks]
\begin{align*}
f(x_0) = 1, ~~~ f'(x_0) = \frac{\pi}{2}, ~~~ x_1 = \frac{\pi}{2} - \frac{2}{\pi}
\end{align*}

\subsection*{Question 4 [20 marks]}
The price of a dividend free stock S follows a geometric Brownian motion with volatility 20\%.
The risk free continuous compounding interest rate is 5\%. The stock price today is 100 dollars.

a) Set up a two step JR (Jarrow, Rudd) binomial tree to price options with 2 year maturity using the stock price as state variable.
You should list and compute the parameters of the tree, and draw out the nodes of the tree with their associated stock price. [10 marks]
\begin{align*}
p = 0.5, ~~~dt = 1, ~~~ \chi_1 = e^{\frac1N rt} = e^{0.05}, ~~~ \chi_2 = e^{\frac1N(2r+\sigma^2)t} = e^{0.14} \\
u = \chi_1 + \sqrt{\chi_2 - \chi_1^2} = 1.2636\\
d = \chi_1 - \sqrt{\chi_2 - \chi_1^2} = 0.8389\\
\end{align*}
\begin{verbatim}
                159.67
        126.36
100             106
        83.89
                70.375
\end{verbatim}
b) Price a 2 year European call option (strike = 95 dollars) on the tree you have set up in a). List the option price at each node of the tree.  [10 marks]
\begin{align*}
df = e^{-0.05} = 0.95123
\end{align*}
\begin{verbatim}
                64.67
        35.99 
20.483          11
        5.2318
                0
\end{verbatim}

\subsection*{Question 5 [20 marks]}
We would like to price an Asian option on a stock $S(t)$ using two moment matching method.
The option's payoff is based on the average of the stock price observed at a sequence of observation times: $A = \frac{1}{n}\sum S(t_i)$.
Two moment matching approximate the distribution of $A$ using a log normal distribution whose first and second moments matches the first and second moments of $A$.
The interest rate is 0\%. The current stock price is $S_0$.
The stock price process follows the Black-Scholes SDE
\begin{align*}
\frac{d(S(t))}{S(t)} = \sigma dW_t
\end{align*}

a) What is formula for $E[A]$ (the first moment of A)? [5 marks]
\begin{align*}
E[A] = S_0
\end{align*}

b) What is formula for $E[A^2]$ (the second moment of A)? [15 marks]
$S = S_0 e^{-\frac12 \sigma^2 t + \sigma W_t}$, assume $t_i < t_j$
\begin{align*}
E[S_i S_j] & = S_0^2 E[e^{-\frac12 \sigma^2 {t_i} + \sigma W_{t_i} -\frac12 \sigma^2 {t_j} + \sigma W_{t_j}}] \\
		& = S_0^2 E[e^{-\sigma^2 {t_i} + 2\sigma W_{t_i} -\frac12 \sigma^2 (t_j-t_i) + \sigma (W_{t_j} - W_{t_i})}] \\
		& = S_0^2 E[e^{-\sigma^2 {t_i} + 2\sigma W_{t_i}}] E[e^{-\frac12 \sigma^2 (t_j-t_i) + \sigma (W_{t_j} - W_{t_i})}] \\
		& = S_0^2 E[e^{-\sigma^2 {t_i} + 2\sigma W_{t_i}}] \\
		& = S_0^2 e^{\sigma^2 t_i}
\end{align*}
\begin{align*}
E[A^2] =  \frac{S_0^2}{n^2} \sum_i^n \sum_j^n e^{\sigma^2 \min(t_i, t_j)}
\end{align*}


\subsection*{Question 6 [15 marks]}
Our underlying asset $X_t$ follows the Ornstein-Uhlenbeck process
\begin{align*}
d X_t = \theta(\mu - X_t) dt + \sigma dW_t
\end{align*}
where $\theta$, $\mu$ and $\sigma$ are all positive constants. The interest rate $r$ is constant as well.
Derive the PDE that the price of a derivative on $X_t$, denoted as $V_t$ should satisfy. 
\begin{align*}
\frac{dB_t}{B_t} & = rdt \\
d{\left(\frac{V_t}{B_t}\right)} & = d{V_t} \frac{1}{B_t} - V_t \frac{1}{B_t^2} dB_t \\
 & = \frac{1}{B_t} (\frac{\partial V}{\partial X} dX + \frac{\partial^2 V}{\partial X^2} \sigma^2 dt + \frac{\partial V}{\partial t} dt) - V_t r dt \frac{1}{B_t} \\
 & = \frac{1}{B_t} (\frac{\partial V}{\partial X} \theta (\mu - X_t) + \frac{\partial^2 V}{\partial X^2} \sigma^2 + \frac{\partial V}{\partial t}  - rV) dt + \frac{1}{B_t} \frac{\partial V}{\partial X} \sigma dW_t
\end{align*}
Since $\frac{V_t}{B_t}$ is martingale, the PDE that $V_t$ satisfies is
\begin{align*}
\frac{\partial V}{\partial X} \theta (\mu - X) + \sigma^2 \frac{\partial^2 V}{\partial X^2} + \frac{\partial V}{\partial t}  - rV = 0
\end{align*}
\end{document}
